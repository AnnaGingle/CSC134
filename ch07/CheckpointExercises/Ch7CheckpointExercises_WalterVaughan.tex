% Checkpoint Exercises for Chapter 7 Homework Assignment
% written by Walter B. Vaughan

\documentclass[11pt]{article}
\author{Walter B. Vaughan\\
        \small CSC 134 -- Section 200 -- Fall 2014\\
        \small Catawba Valley Community College}
\title{Chapter 7 Checkpoint Exercises \\
       \footnotesize{7.1, 7.5, 7.6, 7.8, 7.11, 7.12, 7.13, 7.14, 7.16, 7.18, 7.27, 7.31, 7.32}}
\date{\vspace{-5ex}}
% Section 7.3 No Bounds Checking in C++:  7.1, 7.5, 7.6
% Section 7.6 Parallel Arrays: 7.8, 7.11, 7.12, 7.13
% Section 7.7 Arrays as Function Arguments: 7.14, 7.16, 7.18
% Section 7.11 Vectors: 7.27, 7.31, 7.32

\usepackage{listings}
\usepackage[usenames,dvipsnames]{color}
\usepackage[margin=.5in]{geometry}
\usepackage{indentfirst}

\lstset{frame=tb,
    language=C++,
    columns=flexible,
    basicstyle={\small\ttfamily},
    numbers=left,
    numberstyle=\color{Gray},
    keywordstyle=\color{BlueViolet},
    commentstyle=\color{Gray},
    stringstyle=\color{OliveGreen},
    breaklines=true,
    breakatwhitespace=true,
    showstringspaces=false,
    tabsize=4
}

\begin{document}

\maketitle


\section*{Section 7.3 --- No Bounds Checking in C++}
\begin{description}
    \item[7.1] The arrays would be defined as follows:
    \begin{lstlisting}
int empNums[100];
float payRates[25];
long miles[14];
std::string cityName[26];
double lightYears[1000];
    \end{lstlisting}
    \item[7.5] Array bounds checking is a process which makes sure the user is interacting with an array inside its valid boundaries, so as to avoid unsafe memory access. C++ does not natively perform bounds checking on arrays.
    \item[7.6] \begin{verbatim}
 1
2
3
4
5
    \end{verbatim}
\end{description}

\section*{Section 7.6 --- Parallel Arrays}
\begin{description}
    \item[7.8] The arrays would be defined as follows:
    \begin{lstlisting}
int ages[10] = { 5, 7, 9, 14, 15, 17, 18, 19, 21, 23 };
float temps[7] = { 14.7, 16.3, 18.43, 21.09, 17.9, 18.76, 26.7 };
char alpha[8] = { 'J', 'B', 'L', 'A', '*', '$', 'H', 'M' };
    \end{lstlisting}
    \item[7.11] \texttt{0}
    \item[7.12] \begin{verbatim}
10.00
25.00
32.50
50.00
110.00
    \end{verbatim}
    \item[7.13] \begin{verbatim}
1 18 18
2 4 8
3 27 81
4 52 208
5 100 500
    \end{verbatim}
\end{description}

\section*{Section 7.7 --- Arrays as Function Arguments}
\begin{description}
    \item[7.14] It will not work, because you cannot assign arrays. \lstinline{array2} holds the address of the array, not the values inside the array, and C++ simply does not allow you to modify that address.
    \item[7.16] When used as function arguments, arrays are passed by \emph{pointer}, not value.
    \item[7.18] The completed program is attached as \texttt{7.18.avgarray.cpp}.
\end{description}

\section*{Section 7.11 --- Vectors}
\begin{description}
    \item[7.27] You must use \lstinline{#include <vector>}.
    \item[7.31] \lstinline{gators.push_back(27);}
    \item[7.32] \lstinline{snakes[3] = 12.897;}
\end{description}

\end{document}
