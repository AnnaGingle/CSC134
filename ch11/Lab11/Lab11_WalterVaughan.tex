% Lab 11 writeup
% written by Walter B. Vaughan

\documentclass[11pt]{article}


\author{Walter B. Vaughan\\
        \small CSC 134 -- Section 200 -- Fall 2014\\
        \small Catawba Valley Community College}
\title{Lab 11}
\date{\vspace{-5ex}}


\usepackage{listings}
\usepackage[usenames,dvipsnames]{color}
\usepackage[margin=.5in]{geometry}
\usepackage{indentfirst}


\lstset{frame=tb,
	language=C++,
	columns=flexible,
    keepspaces=true,
	basicstyle={\small\ttfamily},
	numbers=left,
	numberstyle=\color{Gray},
	keywordstyle=\color{BlueViolet},
	commentstyle=\color{Gray},
	stringstyle=\color{OliveGreen},
	breaklines=true,
	breakatwhitespace=true,
	showstringspaces=false,
	tabsize=4
}

\begin{document}

\maketitle


\section*{Pre-Lab Questions}
\begin{enumerate}
    \item The name of a structure is called the \emph{tag}.
    \item An advantage of structures over arrays is that structures allow one to use items of \emph{different} data types, whereas arrays do not.
    \item One structure inside of another structure is an example of a \emph{nested structure}.
    \item The variables declared inside the structure declaration are called the \emph{members} of the structure.
    \item When initializing structure members, if one structure member is left uninitialized, then all the structure members that follow must be \emph{uninitialized}.
    \item A user defined data type is often an \emph{abstract data type}.
    \item Once an array of structures has been defined, you can access an array element using its \emph{index location or subscript}.
    \item The \emph{dot operator (period)} allows the programmer to access structure members.
    \item You may not initialize a structure member in the \emph{definition}.
    \item Like variables, structure members may be used as \emph{function} arguments.
\end{enumerate}
\newpage

\section*{11.1 --- Working with Basic Structures}
    The code was completed and modified as directed in \textbf{Exercises 1 \& 2} and the source codes are included as \texttt{rect\_struct.ex1.cpp} and \texttt{rect\_struct.ex2.cpp}.

\section*{11.2 --- Initializing Structures}
    The code was completed as directed in \textbf{Exercise 1} and the source code is included as \texttt{init\_struct.cpp}. \textbf{Note for the instructor:} The supplied code used the stream modifier \lstinline{setprecision(2)} but the header containing that modifier's definition (\lstinline{<iomanip>}) was not included by default.

\section*{11.3 --- Arrays of Structures}
\begin{description}
    \item[Exercise 1:] The code was completed as directed and the source code is included as \texttt{array\_struct.cpp}.
    \item[Exercise 2:] The snippet of code \lstinline{(index+1)} is used when outputting which taxpayer because array indices are counted starting with \lstinline{0}, but in English, objects are counted starting from \lstinline{1}, so a taxpayer's position (in English) in the array can be thought of as the value of \lstinline{index} plus \lstinline{1}.
\end{description}

\section*{11.4 --- Nested Structures}
\begin{description}
    \item[Exercise 1:] The code was completed as directed and the source code is included as \texttt{nestedRect\_struct.ex1.cpp}. \textbf{Note for the instructor:} The supplied code referred to the \lstinline{area} and \lstinline{perimeter} members through a nested member \lstinline{attributes} but that member did not exist and was not referenced anywhere in the instructions. The appropriate modifications were made to ensure the code compiled correctly.
    \item[Exercise 2:] The code was modified as directed and the source code is included as \texttt{nestedRect\_struct.ex2.cpp}.
    \item[Exercise 3:] The code was modified as directed and the source code is included as \texttt{nestedRect\_struct.ex3.cpp}. I chose to implement the function parameters using pointers.
\end{description}

\section*{11.5 --- Student Generated Code Assignments}
\begin{description}
    \item[Option 1:] The completed program is attached as \texttt{11.5.circ.cpp}. I chose to take a more functional approach to the problem. \lstinline{main()} only has 4 lines to it.
    \item[Option 2:] The completed program is attached as \texttt{11.5.planes.cpp}.
\end{description}


\newpage
{\LARGE Source Code Appendix}
\section*{\texttt{rect\_struct.ex1.cpp}}
\lstinputlisting{rect_struct.ex1.cpp}

\section*{\texttt{rect\_struct.ex2.cpp}}
\lstinputlisting{rect_struct.ex2.cpp}

\section*{\texttt{init\_struct.cpp}}
\lstinputlisting{init_struct.cpp}

\section*{\texttt{array\_struct.cpp}}
\lstinputlisting{array_struct.cpp}

\section*{\texttt{nestedRect\_struct.ex1.cpp}}
\lstinputlisting{nestedRect_struct.ex1.cpp}

\section*{\texttt{nestedRect\_struct.ex2.cpp}}
\lstinputlisting{nestedRect_struct.ex2.cpp}

\section*{\texttt{nestedRect\_struct.ex3.cpp}}
\lstinputlisting{nestedRect_struct.ex3.cpp}

\section*{\texttt{11.5.circ.cpp}}
\lstinputlisting{11.5.circ.cpp}

\section*{\texttt{11.5.planes.cpp}}
\lstinputlisting{11.5.planes.cpp}

\end{document}
