% Lab 5 writeup
% written by Walter B. Vaughan

\documentclass[11pt]{article}
\author{Walter B. Vaughan\\
        \small CSC 134 -- Section 200 -- Fall 2014\\
        \small Catawba Valley Community College}
\title{Lab 6.1}
\date{\vspace{-5ex}}


\usepackage{listings}
\usepackage[usenames,dvipsnames]{color}
\usepackage[margin=.5in]{geometry}
\usepackage{indentfirst}


\lstset{frame=tb,
	language=C++,
	columns=flexible,
	basicstyle={\small\ttfamily},
	numbers=left,
	numberstyle=\color{Gray},
	keywordstyle=\color{BlueViolet},
	commentstyle=\color{Gray},
	stringstyle=\color{OliveGreen},
	breaklines=true,
	breakatwhitespace=true,
	showstringspaces=false,
	tabsize=4
}

\begin{document}

\maketitle


\section*{Pre-Lab Questions}
\begin{enumerate}
	\item The word \texttt{void} precedes the name of every function prototype and heading that does not return a value back to the calling routine.
	\item Pass by \emph{value} indicates that a copy of the actual parameter is placed in the memory location of its corresponding formal parameter.
	\item \emph{Actual} parameters are found in the call to a function.
	\item A prototype must give the \emph{data type} of its formal parameters and may give their \emph{variable names}.
	\item An \emph{ampersand} (\texttt{\&}) after a data type in the function heading and in the prototype indicates that the parameter will be passed by reference.
	\item Functions that do not return a value are often called \emph{procedures} in other programming languages.
	\item Pass by \emph{reference} indicates that the location of an actual parameter, rather than just a copy of its value, is passed to the called function.
	\item A call must have the \emph{variable name} of its actual parameters and must NOT have the \emph{data type} of those parameters.
	\item \emph{Scope} refers to the region of a program where a variable is
"active."
	\item \emph{Formal} parameters are found in the function heading.
\end{enumerate}


\newpage % 6.1 needs its own page for the function
\section*{6.1 --- Functions with No Parameters}

	The file \texttt{proverb.cpp} was modified appropriately and appears as follows: 
	\lstinputlisting{proverb.cpp}


\newpage
\section*{6.2 --- Introduction to Pass by Value}

	\textbf{Exercise 1:} Entering a float value will cause the decimal place to be truncated away, and if what remains is a \texttt{1} stored to \texttt{wordCode}, then the quote will be printed ending with the word \texttt{"party"}, otherwise it will end in \texttt{"country"}. The updated code for this exercise is included as \texttt{newproverb.ex1.cpp}.
	
	\textbf{Exercise 2:} The updated code for this exercise is included as \texttt{newproverb.ex2.cpp}.
	
	\textbf{Exercise 3:} The updated code for this exercise is included as \texttt{newproverb.ex3.cpp}, and appears as follows: \lstinputlisting{newproverb.ex3.cpp}


\newpage
\section*{6.3 --- Introduction to Pass by Reference}

	The appropriate changes were made for \textbf{Exercise 1}, and the program is saved as \texttt{paycheck.ex1.cpp}. The test ouput of the compiled program as directed in \textbf{Exercise 2} is as follows: \begin{verbatim}
Welcome to the Pay Roll Program
************************************************

This program takes two numbers (pay rate & hours)
and  multiplies them to get gross pay 
it then calculates net pay by subtracting 15%
************************************************

Please input the pay per hour
9.50

Please input the number of hours worked
40


The gross pay is $380.00
The net pay is $323.00
We hoped you enjoyed this program
\end{verbatim}

	\textbf{Exercise 3:} The parameters \lstinline{gross} and \lstinline{net} were \emph{pass by reference}.
	
	The changes made as directed in \textbf{Exercise 4} are reflected in the included file \texttt{paycheck.ex4.cpp}, and the compiled program produces the same ouput as previously tested for, as directed in \textbf{Exercise 5}.


\section*{6.4 --- Student Generated Code Assignments}

	\textbf{Option 1:} The completed program compiles and runs, correctly producing the same test output included in the lab directions. The source code is included as \texttt{6.4.swap.cpp}.
	
	\textbf{Option 2:} The program is included as \texttt{6.4.mph.cpp}. The parameters \lstinline{miles} and \lstinline{hours} are passed by value, but the parameter \lstinline{milesPerHour} must be passed by reference.
	
	\textbf{Option 3:} The program is included as \texttt{6.4.grades.cpp}.

\end{document}
