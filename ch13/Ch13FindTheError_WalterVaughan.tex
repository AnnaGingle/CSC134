% Find-the-Error Problems for Chapter 13
% written by Walter B. Vaughan

\documentclass[11pt]{article}
\author{Walter B. Vaughan\\
        \small CSC 134 -- Section 200 -- Fall 2014\\
        \small Catawba Valley Community College}
\title{Chapter 13 Find the Error \\
       \footnotesize{73, 74, 75}}
\date{\vspace{-5ex}}

\usepackage{listings}
\usepackage[usenames,dvipsnames]{color}
\usepackage[margin=.5in]{geometry}
\usepackage{indentfirst}

\lstset{frame=tb,
    language=C++,
    columns=flexible,
    keepspaces=true,
    basicstyle={\small\ttfamily},
    numbers=left,
    numberstyle=\color{Gray},
    keywordstyle=\color{BlueViolet},
    commentstyle=\color{Gray},
    stringstyle=\color{OliveGreen},
    breaklines=true,
    breakatwhitespace=true,
    showstringspaces=false,
    tabsize=4
}

\begin{document}

\maketitle

\begin{description}

    \item[73] Code:
\begin{lstlisting}
class Circle:
{
private
    double centerX;
    double centerY;
    double radius;
public
    setCenter(double, double);
    setRadius(double);
}
\end{lstlisting}
    The above code has a colon after the class name which should be deleted. The \lstinline{private} and \lstinline{public} labels, in contrast, \emph{need} a colon after each. As well, the public functions defined are missing return type identifiers, and the entire class definition should end with a semicolon (which is missing).

\newpage % explanation splits pages otherwise
    \item[74] Code:
\begin{lstlisting}
#include <iostream>
using namespace std;

Class Moon;
{
Private;
    double earthWeight;
    double moonWeight;
Public;
    moonWeight(double ew);
        { earthWeight = ew; moonWeight = earthWeight / 6; }
    double getMoonWeight();
        { return moonWeight; }
}

int main()
{
    double earth;
    cout >> "What is your weight? ";
    cin << earth;
    Moon lunar(earth);
    cout << "On the moon you would weigh "
         <<lunar.getMoonWeight() << endl;
    return 0;
}
\end{lstlisting}
    In the above code, the tokens \lstinline{Class}, \lstinline{Public}, and \lstinline{Private} should all be lowercase. The semicolon after the class name declaration \lstinline{Moon} should be removed, and the semicolons after the \lstinline{public} and \lstinline{private} labels should be turned into colons. It appears that the function \lstinline{moonWeight} was supposed to be a constructor, in which case its name should be changed to \lstinline{Moon}. Otherwise, \lstinline{moonWeight} would need a return type, the declaration of the variable \lstinline{lunar} inside of \lstinline{main()} would need to omit any arguments to the constructor function, and the statement \lstinline{lunar.moonWeight(earth);} would need to be added between lines 21 and 22 to give the object's variables appropriate data. In addition to all this, the class definition for \lstinline{Moon} needs to end with a semicolon, and the semicolons following the first lines of the inline public member function definitions need to be removed.


\newpage
    \item[75] Code:
\begin{lstlisting}
#include <iostream>
using namespace std;

class DumbBell;
{
    int weight;
public:
    void setWeight(int);
};
void setWeight(int w)
{
    weight = w;
}

int main()
{
    DumbBell bar;

    DumbBell(200);
    cout << "The weight is " << bar.weight << endl;
    return 0;
}
\end{lstlisting}
    In the global space of the above code, the semicolon following the name \lstinline{DumbBell} in the class definition should be removed. The function definition of \lstinline{void setWeight(int w)} needs to have the \lstinline{DumbBell} class's namespace prepended to look like this: \lstinline{void DumbBell::setWeight(int w)}.

    Inside of \lstinline{main()}, the call to the \lstinline{DumbBell} constructor with an argument of \lstinline{200} should be a call to \lstinline{setWeight(200)}, and accssing the attribute \lstinline{weight} of \lstinline{bar} is currently impossible without the addition of a class function which will return the value of the private variable \lstinline{weight}. Such a function would be used by putting  \lstinline{bar.getWeight()} in place of \lstinline{bar.weight}.

\end{description}

\end{document}
